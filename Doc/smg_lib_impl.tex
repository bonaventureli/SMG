In the revised example, it is assumed
that {\bf \code{myapp_global}} is a global structure whose type was
reported to SMG by the \SMOBJ directive and that that structure
contains the \code{sm_state} field, and it is also assumed that {\bf
  \code{Iconicize_E}} is the name of an event declared with the \EVENT
directive (or implicitly as part of a \TRANS directive).

The only yet-to-be determined portion of the revised code is the
\code{<event_obj>} specification.  This reference must be for a variable
of a specific type that is useable to sufficiently represent the event
in the state-machine-invoked code.  For the \code{myapp_iconicize}
entry point, there are no parameters and therefore the event is fully
represented just by its presence.  However, other events---such as the
\code{myapp_mouse_b1}---have associated parameters that will need to be
available to the event handling code.  Therefore, the following event
representation structure is defined in \code{myapp} and declared with
the \SMEVT directive:

\begin{verbatim}
typedef union {
    struct {
        int x;
        int y;
    } mouse_coords;
    char keyval;
} myapp_y_event_t;

SM_EVT  myapp_y_event_t *
\end{verbatim}

Now all that's needed is to pass a structure of that type to the SMG
State Machine Event handler routine.  There are a couple of choices as
to where to obtain the structure:

\begin{itemize}

\item From the local stack.  This has the implicit requirement that the
event handling code completely handles the event before returning since
the stack copy will be gone when that return happens and the event entry
point is exited.  This is not usually recommended, but it can be used
in simple situations.

\item From memory allocation.  This is probably the most convenient
method, but it does mean that the event handling might be delayed by
memory allocation time and it also means that the event handling code
must deallocate the event structure when it has been fully handled.

\item From an event object pool.  This uses a pre-allocated pool of
structures which has the advantages (fast allocation) and disadvantages
(resource limitations) of pool structures.  The event handling code
should return the event structure to the pool when finished with it.

\end{itemize}

The example shown in Figure~\ref{figure:event-entry} will use the
simple memory allocation method.  Also note that although {\em all}
events must allocate the structure, different events must initialize
the structure in a different manner and some events not at all,
therefore a global macro can be used to perform the allocation.


\begin{figure*}
\begin{verbatim}
#define INIT_EVT  myapp_y_event_t *evt = (void) malloc(sizeof(myapp_y_event_t)); \
                                  if (!evt) return;

void myapp_mouse_b1(int x, int y) {
    INIT_EVT;
    evt->mouse_coords.x = x;
    evt->mouse_coords.y = y;
    myapp_State_Machine_Event(&myapp_global, evt, Mouse_B1);
}

void myapp_iconicize(void) {
    INIT_EVT;
    myapp_State_Machine_Event(&myapp_global, evt, Iconicize_E);
}
\end{verbatim}
\caption{Event Entry Actions}\label{figure:event-entry}
\end{figure*}

Now that the example is functionally correct again, all that
remains is to recognize that the only variable part is the event
encapsulation/initialization code; SMG can be tasked to create the
invariant parts {\em independently} of the application.   Furthermore the
event object can be defined entirely by the SMG Library since it only
needs to contain the explicit arguments defined by the asynchronous API.
The SMG Library can define \EVENT directives with \code{entry_code}
portions that specify the correct functions and arguments for the event,
and with \code{start_code} portions that perform the per-entry-point
event structure initialization.

The code in
Figures~\ref{figure:y-w-lib-decls},~and~\ref{figure:y-w-lib-events}
show how this is done by implementing the SMG Y-windows Library.

\begin{figure*}
\begin{verbatim}
CODE_{  yw_mouse_b1_decl
    void yw_mouse_b1_e ( int x, int y )
CODE_}
CODE_{  yw_mouse_b2_decl
    void yw_mouse_bw_e ( int y, int x )
CODE_}
CODE    yw_key_entered_decl  void yw_key_entered_e(char keyval)
CODE    yw_expose_decl       void yw_expose_e(void)
CODE    yw_iconicize_decl    void yw_iconicize_e(void)

y_app_vtable_t yw_entrypoints = {
    yw_mouse_b1_e,
    yw_mouse_b2_e,
    yw_key_entered_e,
    yw_expose_e,
    yw_iconicize_e
};
\end{verbatim}
\caption{Y-windows SMG Library declarations}\label{figure:y-w-lib-decls}
\end{figure*}

\begin{figure*}[h]
\begin{verbatim}
typedef union {
    struct {
        int x;
        int y;
    } mouse_coords;
    char keyval;
} yw_event_t;

SM_EVT  yw_event_t *

EVENT  Mouse_B1_E    yw_mouse_b1_decl    yw_mouse_b1_prep
EVENT  Mouse_B2_E    yw_mouse_b2_decl    yw_mouse_b2_prep
EVENT  Key_Entered_E yw_key_entered_decl yw_key_entered_prep
EVENT  Expose_E      yw_expose_decl      yw_noarg_prep
EVENT  Iconicize_E   yw_iconicize_decl   yw_noarg_prep

CODE_{   yw_mouse_b1_prep
// Standard event initialization; defined in CODE tag for _/xxx subst.
#define INIT_EVT  yw_event_t *evt; \
           evt = (void) malloc(sizeof(yw_event_t)); \
           if (!evt) return; \
           _/EVT = evt; \
           _/OBJ = &myapp_global;

    INIT_EVT;
    evt->mouse_coords.x = x;
    evt->mouse_coords.y = y;
CODE_}

CODE_{   yw_mouse_b2_prep
    INIT_EVT;
    evt->mouse_coords.x = x;
    evt->mouse_coords.y = y;
CODE_}

CODE_{   yw_key_entered_prep
    INIT_EVT;
    evt->keyval = keyval;
CODE_}

CODE     yw_noarg_prep    INIT_EVT;
\end{verbatim}
\caption{Y-Windows SMG Library Event Management}\label{figure:y-w-lib-events}
\end{figure*}

%%% Local Variables: 
%%% mode: latex
%%% TeX-master: "smg_guide"
%%% End: 
